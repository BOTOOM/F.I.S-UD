\chapterimage{metodologia.jpg} % Table of contents heading image
\chapter{Metodología}
\section{Manifiesto Agil}
Antes de decir que es Scrum, vale la pena saber que es el manifiesto agil y de que se compone.
\\
\\
El manifiesto agil se compone por 5 valores y 12 Principios:
\\
\\
\subsection{Valores}
\subsubsection{Valorar a las personas y las interacciones entre ellas por sobre los procesos y las herramientas}
Las personas son el principal factor de éxito de un proyecto de software. Es más importante construir un buen equipo que construir el contexto. Muchas veces se comete el error de construir primero el entorno de trabajo y esperar que el equipo se adapte automáticamente. Por el contrario, la agilidad propone crear el equipo y que éste construya su propio entorno y procesos en base a sus necesidades.
\\
\\
\subsubsection{Valorar        el   software       funcionando        por     sobre      la 
	documentación detallada} 
La regla a seguir es "no producir documentos a menos que sean 
necesarios      de    forma     inmediata      para    tomar     un   decisión 
importante". Estos documentos deben ser cortos y centrarse en 
lo esencial. La documentación (diseño, especificación técnica de 
un sistema) no es más que un resultado intermedio y su finalidad 
no    es  dar   valor   en   forma    directa   al   usuario   o   cliente   del 
proyecto. Medir avance en función de resultados intermedios se 
convierte en una simple "ilusión de progreso". 
\\
\\
\subsubsection{Valorar      la   colaboración      con    el   cliente   por    sobre    la 
	negociación de contratos}
Se propone que exista una interacción constante entre el cliente 
y el equipo de desarrollo. Esta mutua colaboración                será la que 
dicte la marcha del proyecto y asegure su éxito. 
\\
\\
\subsubsection{Valorar la respuesta a los cambios por sobre el seguimiento 
	estricto de los planes }
La habilidad de responder a los cambios que puedan surgir a lo 
largo del proyecto (cambios en los requisitos, en la tecnología, 
en el equipo, etc.) determina también su éxito o fracaso. Por lo 
tanto, la planificación no debe ser estricta sino flexible y abierta. 
\\
\\
\subsection{Principios}
Los     valores    anteriores    son    los  pilares    sobre    los  cuales    se 
construyen   los   doce   principios   del   Manifiesto   Ágil.   De   estos 
doce principios, los dos primeros son generales y resumen gran 
parte   del   espíritu   ágil   del   desarrollo   de   software,   mientras   que 
los siguientes son más específicos y orientados al proceso o al 
equipo de desarrollo:\\
\\
\begin{enumerate}[1.]
	\item   Nuestra mayor prioridad es satisfacer al cliente a través 
	de   entregas     tempranas      y  frecuentes     de   software    con 
	valor.

	\item  Aceptar       el   cambio      incluso     en   etapas     tardías    del 
	desarrollo.   Los   procesos   ágiles   aprovechan   los   cambios 
	para darle al cliente ventajas competitivas.
	\item  Entregar software funcionando en formafrecuente , desde 
	un   par   de   semanas   a   un   par   de   meses,   prefiriendo   el 
	periodo de tiempo más corto.
	\item   Expertos     del   negocio    y  desarrolladores      deben    trabajar 
	juntos diariamente durante la ejecución del proyecto.
	\item   Construir      proyectos     en   torno    a  personas     motivadas, 
	generándoles       el   ambiente      necesario,     atendiendo      sus 
	necesidades y confiando en que ellos van a poder hacer 
	el trabajo.
	\item  La    manera     más    eficiente   y   efectiva    de  compartir      la 
	información       dentro   de   un   equipo    de   desarrollo    es   la 
	conversación cara a cara.
	\item  El   software      funcionando      es   la   principal    métrica    de 
	progreso.
	\item  Los procesos ágiles promueven el desarrollo sostenible. 
	Los     sponsors,    desarrolladores      y  usuarios    deben    poder 
	mantener un ritmo constante indefinidamente.
	\item  La   atención   continua   a   la   excelencia   técnica   y   buenos 
	diseños incrementan la agilidad.
	\item  La   simplicidad    –el    arte  de   maximizar      la  cantidad    de 
	trabajo no hecho- es esencial.
	\item  Las    mejores     arquitecturas,     requerimientos       y   diseños 
	emergen de equipos auto-organizados.
	\item  A   intervalos   regulares,   el   equipo   reflexiona   acerca   de 
	cómo convertirse en más efectivos, luego mejora y ajusta 
	su comportamiento adecuadamente.
\end{enumerate}

\section{Scrum}
Scrum es una de las metodologia agiles mas conocidas al igual que XP, es una herramienta en la cual se tienen varios grupos de trabajo enfocados en diversas tareas con un unico objetivo, en el cual se espera generar resultados y/o entregas al cliente en cortos periodos de tiempo.
\\
\\
El equipo de trabajo en esta metodologia esta apoyado por 2 roles:  El ScrumMaster y el Product Owner. el ScrumMaster o tambien consederado Coach es aquel que vela por que se utilice Scrum , por remover las impedimentos y da asistencia al equipo para que logre el mayor nivel de rendimiento  posible.  El   Product   Owner   es 
quien     representa      al  negocio,  stakeholders ( trabajadores, organizaciones sociales, accionistas y proveedores, entre muchos otros actores clave ),         cliente    y  usuarios 
finales. 
\subsection{En que se basa}

Esta basada en tres pilares:
\\
\\
Transparencia: Todos los implicados tienen conocimiento de qué ocurre y en el proyecto y cómo ocurre. Esto hace que haya un entendimiento “común” del proyecto, una visión global.
\\
\\
Inspección: Los miembros del equipo Scrum frecuentemente inspeccionan el progreso para detectar posibles problemas. La inspección no es un examen diario, sino una forma de saber que el trabajo fluye y que el equipo funciona de manera auto-organizada.
\\
\\
Adaptación: Cuando hay algo que cambiar, el equipo se ajusta para conseguir el objetivo del sprint. Esta es la clave para conseguir éxito en proyectos complejos, donde los requisitos son cambiantes o poco definidos y en donde la adaptación, la innovación, la complejidad y flexibilidad son fundamentales.


