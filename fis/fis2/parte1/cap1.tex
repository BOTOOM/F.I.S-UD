\chapterimage{problema.jpg} % Table of contents heading image
\chapter{Problema}
\section{Problema}

Dificultad que tienen los estudiantes para aprovechar el apoyo alimentario que ofrece la Universidad Distrital Francisco José de Caldas.



\section{Objetivos}

Crear una aplicación que permita agilizar el proceso de entrega de almuerzo
Mejorar el servicio de apoyo alimentario 
Automatizar el proceso registro de asistencia de usuarios
\\
\\
\textbf{Pregunta: ¿Cómo agilizar el proceso de entrega de almuerzos en la Universidad ?}

\section{Hipotesis:}

Una estrategia para agilizar el proceso de entrega de almuerzos en la universidad es la construcción de un aplicativo web que permita la entrega de papeles, organización de horarios y la verificación de asistencia al apoyo alimentario.
\\
Es necesario que a través del aplicativo se pueda organizar horarios ya que si solo se gestiona la asistencia y entrega de papeles no habra una solucion al problema a largo plazo pero si gestionan los horarios estudiantiles se pueden crear estrategias para evitar el aglutinamiento de personas a la hora de recibir el almuerzo.
\\
\clearpage
\section{Justificación:}
En la Universidad Distrital Francisco José de Caldas existe un serio problema de gestión en la distribución de almuerzo a los estudiante:  la lentitud en la entrega ha ocasionado que los estudiantes no puedan aprovechar el servicio sin correr el riesgo de comprometer la llegada a sus clases. Muchos estudiantes necesitan el apoyo alimentario pero el llegar tarde a clase da lugar incidentes entre profesores y alumnos lo que  puede afectar, a su vez, el desempeño académico