\chapter{Análisis}

\section{Introducción}


\section{Diagrama de Casos de Uso}
El diagrama de casos de uso es una forma de representar los requerimientos de un sistema, cada caso de uso reune una serie de requisitos basandose en diferemtes funciones o tareas.
\\
\\
Actor: Es una agrupacion uniforme de personas, sistemas o maquinas que interactuan con el sistema que estamos construyendo.   
Caso de Uso: Son secuencias de interacciones entre actores y un sistema que usan un servicio. 

\begin{table}[]
	\centering
	\caption{Caso de uso 1}
	\label{my-label}
	\begin{tabular}{|l|l|}
		\hline
		Nombre      & Recibir apoyo alimentario                                 \\ \hline
		Actores     & Estudiantes                                               \\ \hline
		Escenario   &                                                           \\ \hline
		Primario    & Recibir el apoyo alimentario                              \\ \hline
		Secundario  & no coinciden horarios,                                    \\ \hline
		Excepciones & no estar registrado, no es estudiante, se acabo la comida \\ \hline
	\end{tabular}
\end{table}

\begin{table}[]
	\centering
	\caption{Caso de uso 2}
	\label{my-label}
	\begin{tabular}{|l|l|}
		\hline
		Nombre      & Realizar Inscripcion                                                                                                                                                                   \\ \hline
		Actores     & Estudiante                                                                                                                                                                             \\ \hline
		Escenario   &                                                                                                                                                                                        \\ \hline
		Primario    & Inscribiruna solicitud al apoyo alimentario                                                                                                                                            \\ \hline
		Secundario  & Olvido datos de usuario                                                                                                                                                                \\ \hline
		Excepciones & \begin{tabular}[c]{@{}l@{}}No es estudiante o es un egresado, tener el semestre aplazado,\\  error con el servidor, error con el cliente\\ No hay convocatoria disponible\end{tabular} \\ \hline
	\end{tabular}
\end{table}

\begin{table}[]
	\centering
	\caption{Caso de uso 3}
	\label{my-label}
	\begin{tabular}{|l|l|}
		\hline
		Nombre      & Realizar Inscripcion                                                                                                                                    \\ \hline
		Actores     & Cordinador Beinestar                                                                                                                                    \\ \hline
		Escenario   &                                                                                                                                                         \\ \hline
		Primario    & Generar convocatoria apoyo alimentario                                                                                                                  \\ \hline
		Secundario  & Olvido datos de usuario                                                                                                                                 \\ \hline
		Excepciones & \begin{tabular}[c]{@{}l@{}}No hay acceso a la plataforma, No tiene los permisos necesarios,\\  error con el servidor, error con el cliente\end{tabular} \\ \hline
	\end{tabular}
\end{table}
\clearpage
% Please add the following required packages to your document preamble:
% \usepackage[table,xcdraw]{xcolor}
% If you use beamer only pass "xcolor=table" option, i.e. \documentclass[xcolor=table]{beamer}

\newpage

\section{Interacciones}

\newpage

\subsection{Diagrama de Secuencia}

\newpage

\subsection{Diagrama de Comunicación}

\newpage

\subsection{Diagrama de Temporización}

\newpage

\section{Diagramas de Actividades}

\newpage

\section{Diagramas de Actividades}

\newpage

\section{Diagramas de Workflow}

\newpage

\section{Diagramas de Descripción de la Interacción}

\newpage
