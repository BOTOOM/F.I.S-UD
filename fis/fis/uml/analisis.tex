\chapter{Análisis}

\section{Introducción}

\newpage

\section{Diagrama de Casos de Uso}
El diagrama de casos de uso es una forma de representar los requerimientos de un sistema, cada caso de uso reune una serie de requisitos basandose en diferemtes funciones o tareas.
\\
\\
Actor: Es una agrupacion uniforme de personas, sistemas o maquinas que interactuan con el sistema que estamos construyendo.   
Caso de Uso: Son secuencias de interacciones entre actores y un sistema que usan un servicio. 

\\
\begin{table}[]
\centering
\caption{Caso de Uso 2}
\label{my-label}
\begin{tabular}{lllll}
Nombre      & Realizar Inscripcion                                                                                                                                                                &  &  &  \\
Actores     & Estudiante                                                                                                                                                                          &  &  &  \\
Escenario   &                                                                                                                                                                                     &  &  &  \\
Primario    & Inscribir una solicitud al apoyo alimentario                                                                                                                                        &  &  &  \\
Secundario  & Olvido datos de usuario                                                                                                                                                             &  &  &  \\
Excepciones & \begin{tabular}[c]{@{}l@{}}No es estudiante o es un egresado, tener el semestre aplazado, error con el servidor, error con el cliente\\ No hay convocatoria disponible\end{tabular} &  &  & 
\end{tabular}
\end{table}


\newpage

\section{Interacciones}

\newpage

\subsection{Diagrama de Secuencia}

\newpage

\subsection{Diagrama de Comunicación}

\newpage

\subsection{Diagrama de Temporización}

\newpage

\section{Diagramas de Actividades}

\newpage

\section{Diagramas de Actividades}

\newpage

\section{Diagramas de Workflow}

\newpage

\section{Diagramas de Descripción de la Interacción}

\newpage
