\chapter{Metodología}
\section{Scrum}
Scrum es una de las metodologia agiles mas conocidas al igual que XP, es una herramienta en la cual se tienen varios grupos de trabajo enfocados en diversas tareas con un unico objetivo, en el cual se espera generar resultados y/o entregas al cliente en cortos periodos de tiempo.
\\
\subsection{En que se basa}

Esta basada en tres pilares:
\\
\\
Transparencia: Todos los implicados tienen conocimiento de qué ocurre y en el proyecto y cómo ocurre. Esto hace que haya un entendimiento “común” del proyecto, una visión global.
\\
\\
Inspección: Los miembros del equipo Scrum frecuentemente inspeccionan el progreso para detectar posibles problemas. La inspección no es un examen diario, sino una forma de saber que el trabajo fluye y que el equipo funciona de manera auto-organizada.
\\
\\
Adaptación: Cuando hay algo que cambiar, el equipo se ajusta para conseguir el objetivo del sprint. Esta es la clave para conseguir éxito en proyectos complejos, donde los requisitos son cambiantes o poco definidos y en donde la adaptación, la innovación, la complejidad y flexibilidad son fundamentales.